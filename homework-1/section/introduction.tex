\section{Introduction}
\label{sec:introduction}
%
%Introduce the context, motivations, and goals of your project.
%
%The paper is organized as follows: Section~\ref{sec:methodology} describes our approach; Section~\ref{sec:setup} explains our experimental setup; Section~\ref{sec:results} discusses our main findings; finally, Section~\ref{sec:conclusion} draws some conclusions and outlooks for future work.
%introduzione: cos'è CLEF LongEval, obiettivo dell'iniziativa, cosa dobbiamo produrre e quale sarebbe la produzione ideale: obiettivo del nostro lavoro.
%Struttura del documento: sez1 abcdef, sez2 abdef...

Search Engines (SE) are used daily by every person in the world. In particular, the main application of SE is the web search consisting in the retrieval of documents (web pages) in the web. Generally, people expect to provide a sentence as input (a query, representing their interest or information need) and to get back a list of results that are highly related (relevant) to that piece of text.
\par
This document summarizes the information retrieval systems developed by the team DARDS as part of the Search Engines 2022/2023 course, which is held at the University of Padua, in order to address the task 1 "Retrieval" of the "LongEval" lab proposed by CLEF 2023. The goal of the lab, hence of the systems, is to retrieve the most relevant documents given a query (a short sentence representing what the user would use to look for its information need). Furthermore, the lab aims at understand the time persistence and reliability of the developed systems by testing them using some data sampled at different times.

\par
The paper is organized as follows: Section~\ref{sec:methodology} describes our approach; Section~\ref{sec:setup} explains our experimental setup; Section~\ref{sec:results} discusses our main findings; finally, Section~\ref{sec:conclusion} draws some conclusions and outlooks for future work.
